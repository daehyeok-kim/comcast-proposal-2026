\section{Introduction}
\label{sec:intro}

Latency-critical edge applications, such as distributed AI inference and training, XR offloading, real-time video analytics, targeted advertising, and cost-effective distribution of live events, are emerging as primary drivers of demand across network-edge platforms, including CDNs, 5G MEC deployments, and ISP-operated edge infrastructures.
These applications impose strict end-to-end latency SLOs, often in the tens of milliseconds, and are highly sensitive to tail latency.
However, today's edge infrastructures routinely fail to deliver predictable performance under realistic multi-tenant workloads.

Our recent work~\cite{nsdi26-smec} provides empirical evidence that SLO-aware resource management is both necessary and effective at the network edge.
Through extensive measurements on commercial MEC deployments across multiple cities and operators, we show that today's edge platforms, despite low median latency, suffer from severe tail latency and frequent SLO violations due to network and compute contention.
Moreover, the work demonstrates that lightweight, deadline-aware scheduling can improve SLO satisfaction to over 90\% while reducing tail latency by orders of magnitude under realistic multi-tenant workloads.
Motivated by these findings, this proposal addresses a broader question of direct relevance to Comcast:

\begin{taskbox}
\textbf{Question:} How can edge service providers deliver predictable, SLO-driven services across heterogeneous access networks and increasingly tight latency budgets, without requiring invasive coordination or over-provisioning?
\end{taskbox}

We propose to develop a network-agnostic, SLO-aware edge computing framework that generalizes across CDN, MEC, and ISP edge deployments.
The proposed framework is designed to support both today's latency-sensitive services (\eg live video analytics and edge AI inference) and emerging applications with tighter constraints, such as cloud AR/VR and interactive XR offloading.
The project will deliver a deployable, open-source prototype and concrete design guidance for diverse edge platforms.