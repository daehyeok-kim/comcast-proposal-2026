\section{Overview of Proposed Research}
\label{sec:overview}

To address these challenges, we propose a network-agnostic, SLO-aware edge computing framework applicable to diverse edge platforms including ISP-operated edge, CDN edge, and MEC deployments.
The framework will enable predictable, low-latency services across heterogeneous access networks without requiring tight coordination between network and compute layers.
This proposal pursues three research goals:
(1) design network-agnostic mechanisms for SLO awareness that work across cable, Wi-Fi, cellular, and wired access networks,
(2) extend support to emerging applications with sub-50 ms latency constraints, and
(3) ensure practical deployability through modular, incrementally adoptable components.
We translate these goals into three tightly scoped research thrusts:

\mypara{Thrust 1: Network-Agnostic Time-Budget Estimation.}
Build on SMEC's lightweight time-budget estimation while removing reliance on 5G-specific control-plane signals.
The thrust will develop passive estimation techniques that work across diverse access networks (cable, Wi-Fi, cellular, wired) using only end-to-end observable signals at the edge server.

\mypara{Thrust 2: Deadline-Aware Scheduling for Heterogeneous Edge Resources.}
Develop scheduling policies for CPU and GPU resources that prioritize requests based on remaining time budgets while maintaining multi-tenant isolation.
The thrust will extend support to emerging applications with sub-100 ms SLOs, including cloud AR/VR and interactive XR offloading.

\mypara{Thrust 3: Overload Control and SLO-Centric Fairness.}
Design admission control and early-drop mechanisms that maximize SLO satisfaction during congestion.
The thrust will implement SLO-centric fairness where protection is based on request urgency and service-tier policies rather than uniform resource allocation.

\mypara{System Architecture.}
We envision the framework consisting of two modular components:
(1) a Time Budget Estimator deployed at edge servers that estimates per-request time budgets using passive observation of network timing patterns, and
(2) an Edge Resource Manager that implements deadline-aware scheduling for CPU and GPU resources, admission control, and early-drop policies, with integrated policy interfaces that enable operators to configure service tiers and fairness policies without modifying application code.
All components will operate without tight coupling to access networks, enabling incremental deployment on existing edge infrastructure across ISPs, CDNs, and other edge platforms.

Section~\ref{sec:research} details the technical approach, challenges, and expected contributions for each thrust.